\documentclass[12pt,a4paper,oneside]{report}
\usepackage[spanish]{babel}
\selectlanguage{spanish}
\usepackage[T1]{fontenc}
\usepackage{times}
\usepackage[utf8]{inputenc}
\usepackage{amsmath}
\usepackage{graphicx}
\usepackage{multicol}
\usepackage{longtable}
\usepackage[refpages]{gloss}
\usepackage{float}
\usepackage{anysize}
\usepackage{bigstrut}
\usepackage{appendix}
\usepackage{lscape} 
\usepackage{pdflscape}
\usepackage{multirow}
\usepackage{listings}
\usepackage{color}
\usepackage{setspace}
\usepackage{enumerate} 
\usepackage{ragged2e}
\usepackage[utf8]{inputenc}
\usepackage{comment}
\usepackage{pslatex}
\usepackage{apacite} 
\usepackage{fixltx2e}
\usepackage{caption}
\usepackage{setspace} 
\captionsetup[table]{skip=10pt}
\bibliographystyle{apacite}
\begin{document}
\renewcommand{\BOthers}[1]{et al.\hbox{}}
\renewcommand\bibname{Bibliografía}
%----------------------------------------------------------------------------------------
%	CONFIGURACION
%----------------------------------------------------------------------------------------
\marginsize{3.0cm}{3.0cm}{3.0cm}{3.0cm}
\renewcommand*{\contentsname}{Tabla de contenidos}
\renewcommand*{\listtablename}{Índice de tablas}
\renewcommand*{\listfigurename}{Índice de figuras}
\renewcommand{\baselinestretch}{1.0}
\renewcommand{\appendixname}{Anexos}
\renewcommand{\appendixtocname}{Anexos}
\renewcommand{\appendixpagename}{Anexos}
\renewcommand{\thetable}{\arabic{chapter}.\arabic{table}}
\renewcommand*{\tablename}{Tabla}
\renewcommand*{\chaptername}{Capítulo}
\renewcommand*{\thechapter}{\Roman{chapter}}
\renewcommand{\thesection}{\arabic{chapter}.\arabic{section}}
\renewcommand{\figurename}{Figura}
\renewcommand{\thefigure}{\arabic{chapter}.\arabic{figure}}
\renewcommand{\theequation}{\arabic{chapter}.\arabic{equation}}

%----------------------------------------------------------------------------------------
%	Carátula
%----------------------------------------------------------------------------------------
\begin{titlepage}
\begin{center}
 {\Large \bf UNIVERSIDAD CATÓLICA DE SANTA MARÍA}\\
  \vspace{8mm} 
  {\Large \bf Facultad de Ciencias e Ingenieria Fisicas y Formales}\\
  \vspace{8mm}
  {\Large \bf Escuela Profesional de Ingeniería de Sistemas}\\
 \begin{figure}[H]
    \centering
    \includegraphics[width=7cm]{imagenes/logo de la catolica.png}
\end{figure}
\title{MEJORA DEL PROCESO DE RESERVACIONES DE RUTAS TURÍSTICAS DE UNA HOTELERA MEDIANTE LA UTILIZACIÓN DE UN CHATBOT} % titulo de tu tesis para latex
{\Large \bf }
\vspace{1cm}
{\bf DIANTE LA UTILIZACIÓN DE UN CHATBOT}\\[1.0cm]
\begin{flushright}
{\bf AUTOR}\\[0.5cm]
{Huanca Paucar Renato Denilzón}\\[0.5cm] % nombres del autor o autores [1.0cm]
{Sanchez Coila Arnold Bryan}\\[1.0cm]

{\bf ASESOR(A)}\\[0.5 cm] 
{Dra. Castro Gutierrrez Eveling Gloria}\\[0.5 cm] % nombre del asesor
\end{flushright}
\vspace{1cm}
{Arequipa - Perú}\\[0.5cm]
{2022}
\end{center}
\end{titlepage}

\pagenumbering{roman}
%----------------------------------------------------------------------------------------
%	Resumen
%----------------------------------------------------------------------------------------

\chapter*{\centering \large Resumen} 
\addcontentsline{toc}{chapter}{Resumen} % si queremos que aparezca en el índice
\markboth{Resumen}{Resumen} % encabezado
{Actualmente los sistemas de informaci ́on pueden brindar un gran apoyo en el proceso de la
reserva tur ́ıstica en un entorno hotelero, sin embargo, solo teniendo dichos sistemas aun hace falta
personal para que pueda atender dudas o inquietudes que sean muy espec ́ıficas por parte de los
usuarios. Los sistemas de informaci ́on son las herramientas adecuadas para gestionar y brindar
informaci ́on a los turistas, ya sean nacionales o internacionales, y as ́ı satisfacer sus necesidades. El
turismo siempre ha sido una de las fuentes de ingresos m ́as importantes para un pa ́ıs o una ciudad,
por eso es tan importante contar con una adecuada gesti ́on tur ́ıstica en nuestra ciudad, pues se
espera que sea la satisfacci ́on de los turistas. El enfoque de este documento se basa en el aporte
de una inteligencia artificial que se encargue del procesamiento del lenguaje natural para que este
pueda interactuar con el usuario y lo apoye en todo el ciclo de realizaci ́on de la reserva tur ́ıstica
en el hotel.}\\[0.5cm]


%----------------------------------------------------------------------------------------
%	Abstract
%----------------------------------------------------------------------------------------

\chapter*{\centering \large Abstract} 
\addcontentsline{toc}{chapter}{Resumen} % si queremos que aparezca en el índice
\markboth{Resumen}{Resumen} % encabezado
{Abstract in english}\\[0.5cm]

%----------------------------------------------------------------------------------------
%	Introducción
%----------------------------------------------------------------------------------------
\chapter*{\centering \large Introducción} 
\addcontentsline{toc}{chapter}{Resumen} % si queremos que aparezca en el índice
\markboth{Resumen}{Resumen} % encabezado
{Escribir resumen aquí}\\[0.5cm]
%----------------------------------------------------------------------------------------

%	TABLA DE CONTENIDOS
%---------------------------------------------------------------------------------------
\cleardoublepage
\addcontentsline{toc}{chapter}{\contentsname}
\tableofcontents \newpage
\addcontentsline{toc}{chapter}{\listfigurename}
\listoffigures \newpage
\addcontentsline{toc}{chapter}{\listtablename}
\listoftables \newpage
\makegloss
\newpage
\pagenumbering{arabic}

%----------------------------------------------------------------------------------------
%	Capítulo 1
%----------------------------------------------------------------------------------------
\doublespacing
\chapter{Título de capítulo 1}
\section{Título de sección}
\subsection{Título de subsección}
\subsubsection{Título de subsubsección}
\begin{spacing}{1.5}
{Ejemplo de cuerpo y ejemplo de imagen en l \ref{fig:PLACEHOLDER}. }
\begin{figure}[H]
    \centering
    \includegraphics[width=0.5\textwidth]{imagenes/640x360.png}
    \caption{Ejemplo de imagen.}
    \label{fig:PLACEHOLDER}
\end{figure}

\begin{equation} \label{eq:1}
    S=\sum_iP_ilog(P_i)
\end{equation}


\end{spacing}
%----------------------------------------------------------------------------------------
%	BIBLIOGRAFIA
%----------------------------------------------------------------------------------------

\bibliography{library}


%----------------------------------------------------------------------------------------
%	Anexos
%----------------------------------------------------------------------------------------
\appendix
\renewcommand{\appendixname}{Anexo}%
\chapter*{\appendixname}
\phantomsection
\end{document} 
